\chapter{Troubleshooting}\label{ch:troubleshooting}

\section{Acceptable errors}

\subsection{Base}
\subsubsection{Hector Pose Estimation}
\texttt{Skipping XML Document /opt/ros/kinetic/share/\\
hector\_pose\_estimation/hector\_pose\_estimation\_nodelets.xml which had no Root Element.  This likely means the XML is malformed or missing.} \\

This will show when launching the base. We don't use hector pose estimation, so this does not affect anything.

\subsection{Arduino}
\subsubsection{Lost connection}
\texttt{Lost sync with device, restarting...} \\

This shows up periodically, for a yet-to-be-determined reason. It does not seem to negatively affect input/output from the arduino(s).

\section{Connectivity and Hardware Issues}
\subsection{Dynamixel driver crashes}
This is almost always a conflict with USB devices. Theoretically the \texttt{udev} rules should eliminate the issue (these exist in \texttt{poli\_launch/config}). \\
Unfortunately the only answer currently is to un/plug the USB2Dynamixel adapter and restart the driver.

\subsection{Gripper not connected}
Exact error: `Unable to connect, please check the port and address used' \\

Whats wrong: The gripper is not connected to the network. Try power cycling the arm. Unfortunately something internal to the gripper is preventing it from connecting. This happens with DHCP and a static IP on the gripper.

\subsection{Gripper connected but does not respond to position goals}
Exact error: TODO \\

Whats wrong: The gripper has likely hit a fault as a protection mechanism. 
To continue with normal operation, the fault needs to be acknowledged. 
An example of how this can be done is in Example \ref{ex:gripper_fault_ack}.

Note: Due to a limitation in the design of the gripper-arm circuit pair, the gripper frequently hits a Fast Stop error. A solution for this is in the works.


\subsection{TF tree broken / no laser scans output in RViz / localization fails}
Exact error: `No laser scan received (and thus no pose updates have been published) ' \\

Whats wrong: The scan merger node was launched before the lidar drivers were fully up, or it wasn't launched at all. Running \texttt{rosrun poli\_launch scan\_merger.launch} in a separate terminal solves this.


\section{Issues with existing code bases}

More specific issues that arise coming from other packages or libraries should be opened as issues on the respective issue tracker. \\

Issues or additions to the PoliV2 repository should be opened as a ticket at \href{https://github.com/si-machines/poli2/issues}{https://github.com/si-machines/poli2/issues} or as a Pull Request.
  